\section{Data}
1. The data for this project will be collected from the University of North Carolina's Canvas API (Will use Murtadha's Account) to fetch the number of students in the Computer Vision course. This information will be stored to be references to give a percentage of students attending.\\
2. The dataset that will be used to train our model is the \href{http://vis-www.cs.umass.edu/lfw/}{\underline{Faces in the Wild (LFW)}} dataset. This dataset contains more than 13,000 images of faces collected from the web. Each face has been labeled with the name of the person pictured. 1680 of the people pictured have two or more distinct photos in the dataset. The only constraint on these faces is that they were detected by the Viola{-}Jones face detector.\\
3. Images of the students in the class will be collected using a webcam. The images will be taken in the classroom and will be used to train the model to recognize the students in the class.\\
The dataset used to test the model consists of photographs taken by group members during each class session. These photos were captured with the consent of the participants to record attendance for the day. Additionally, a small amount subset of a wild dataset will be incorporated as supplementary learning material.\\
Our approach involves training the model on a portion of the wild dataset to help it recognize and understand human faces. Subsequently, we will use our own images as test data to assess whether the model has been trained effectively. The primary objective is to determine the number of human faces in these images by performing face detection.\\
The dataset will be split into two parts: training and testing. The training dataset will be used to train the model to recognize the faces of the students in the class. The testing dataset will be used to test the model's accuracy.\\
The training dataset will consist of 80\% of the images in the dataset. The testing dataset will consist of the remaining 20\% of the images in the dataset.\\


